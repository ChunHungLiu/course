\documentclass[pdftex]{beamer}
\mode<presentation>
{
  \usetheme{default}
  \useoutertheme{infolines}
}

\usepackage[english]{babel}
\usepackage[latin1]{inputenc}
\usepackage{times}
\usepackage[T1]{fontenc}
\usepackage{fancyvrb}
\usepackage{listings}
\begin{document}
\lstset{language=C, escapeinside={(*@}{@*)}, numbers=left,
  basicstyle=\tiny, showspaces=false, showtabs=false}

\title{A Look Inside FreeBSD with DTrace}
\subtitle{Beneath the Filesystem}
\author[shortname]{George V. Neville-Neil \and Robert N. M. Watson}

\begin{frame}
  \frametitle{GEOM Oveview}
  \begin{itemize}
  \item Framework for transforming storage requests
  \item Object Oriented
  \item Shuttles I/O requests between filesystems and devices
  \end{itemize}
\end{frame}

\begin{frame}
  \frametitle{GEOM Layers}
  \begin{description}
  \item[MBR] Master Boot Record
  \item[BSD] BSD Slice
  \item[ELI] Encryption
  \item[MIRROR] Disk mirroring
  \item[JOURNAL] Journaling
  \item[RAID] Software RAID
\end{description}
\end{frame}

\begin{frame}
  \frametitle{A Visual Example}
  
\end{frame}

\begin{frame}
  \frametitle{Performing I/O}
sudo dtrace -n ::g\_io\*:entry  

This gets us the most likely 8 routines we care about.

\end{frame}

\begin{frame}
  \frametitle{Read Bandwidth}
  
\end{frame}

\begin{frame}
  \frametitle{Write Bandwidth}
  
\end{frame}

\begin{frame}
  \frametitle{Read Latency}
  
\end{frame}

\begin{frame}
  \frametitle{Write Latency}
  
\end{frame}

\begin{frame}
  \frametitle{Layering Costs}
  
\end{frame}

% Look at using the consumer

\end{document}

%%% Local Variables:
%%% mode: latex
%%% TeX-master: t
%%% End:
