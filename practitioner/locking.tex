\documentclass[pdftex]{beamer}
\usetheme{metropolis}

\usepackage[english]{babel}
\usepackage[latin1]{inputenc}
\usepackage{times}
\usepackage[T1]{fontenc}
\usepackage{fancyvrb}
\usepackage{listings}
\begin{document}
\lstset{language=C, escapeinside={(*@}{@*)}, numbers=left,
  basicstyle=\tiny, showspaces=false, showtabs=false, showstringspaces=false}

\title{A Look Inside FreeBSD with DTrace}
\subtitle{Locking}
\author[shortname]{George V. Neville-Neil \and Robert N. M. Watson}

\begin{frame}
  \frametitle{Locking Overview}
  \begin{itemize}
  \item Multiple hardware threads
  \item Protection for shared state
  \item Locking required for MP and multi-core systems
  \end{itemize}
\end{frame}

\begin{frame}
  \frametitle{Types of Locks}
  \begin{description}
  \item[mutex] Most basic lock
  \item[r/w lock] Reader/writer lock
  \item[sx lock] Shared/exclusive lock
  \end{description}
\end{frame}

\begin{frame}[fragile]
  \frametitle{Who is locking?}
\begin{verbatim}
dtrace -n 'lockstat::: { @locks[execname] = count();} '
  nfsd                                                             16
  sudo                                                             35
  sendmail                                                         48
  fdc0                                                             56
  syncer                                                           68
  bufdaemon                                                       272
  sshd                                                            293
  kernel                                                          320
  vmdaemon                                                        670
...
\end{verbatim}
\end{frame}

\begin{frame}
  \frametitle{Lock Provider}
  \begin{itemize}
  \item All lock types have similar probes
  \item 29 total probes in all
  \end{itemize}
  \begin{description}
  \item[acquire] Code grabbed the lock
  \item[block] Another thread was blocked
  \item[spin] Thread is spinning on the lock
  \item[release] Lock is released
  \end{description}
\end{frame}

\begin{frame}
  \frametitle{Lock Provider Arguments}
  \begin{description}
  \item[arg0] Address of lock object
  \item[arg1] Lock time or spin count
  \end{description}
\end{frame}

\begin{frame}[fragile]
  \frametitle{Sum kernel adaptive lock block time by process name}
\begin{verbatim}
dtrace -n 'lockstat:::adaptive-block { @[execname] = sum(arg1); }'
\end{verbatim}
\end{frame}

\begin{frame}[fragile]
\frametitle{Summarize adaptive lock block time distribution by process name}
\begin{verbatim}
dtrace -n 'lockstat:::adaptive-block { @[execname] = quantize(arg1); }'
\end{verbatim}
\end{frame}

\begin{frame}[fragile]
\frametitle{Sum kernel adaptive lock block time by kernel stack trace}
\begin{verbatim}
dtrace -n 'lockstat:::adaptive-block { @[stack()] = sum(arg1); }'
\end{verbatim}
\end{frame}

\begin{frame}[fragile]
\frametitle{Sum kernel adaptive lock block time by lock name}
\begin{verbatim}
dtrace -n 'lockstat:::adaptive-block { @[arg0] = sum(arg1); } END { printa("%40a %@16d ns\n", @); }'
\end{verbatim}  
\end{frame}

\begin{frame}[fragile]
\frametitle{Sum kernel adaptive lock block time by calling function}
\begin{verbatim}
dtrace -n 'lockstat:::adaptive-block { @[caller] = sum(arg1); } END { printa("%40a %@16d ns\n", @); }'
\end{verbatim}
\end{frame}

\begin{frame}
  \frametitle{Locking Lab Exercises}
  \begin{itemize}
  \item Adapt each on-liner to one other lock type
  \item Create a one-liner to track lock acquisition
  \item Track spin locks
  \item Write a one-liner to show where locks are acquired
  \end{itemize}
\end{frame}

\end{document}

%%% Local Variables:
%%% mode: latex
%%% TeX-master: t
%%% End:
