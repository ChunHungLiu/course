% This syllabus template was created by:
% Brian R. Hall
% Assistant Professor, Champlain College
% www.brianrhall.net

% Document settings
\documentclass[11pt]{article}
\usepackage[margin=1in]{geometry}
\usepackage[pdftex]{graphicx}
\usepackage{multirow}
\usepackage{setspace}
\pagestyle{plain}
\setlength\parindent{0pt}

\begin{document}

% Course information
\begin{tabular}{ l l }
%  \multirow{3}{*}{\includegraphics[height=1.25in,width=1in]{logo_blank.png}} & \LARGE Unknown \\\\
  & \LARGE Introduction to Operating Systems with Tracing \\\\
  & \LARGE Day(s), Time, Place \\\\
\end{tabular}
\vspace{10mm}

% Professor information
\begin{tabular}{ l l }
%  \multirow{6}{*}{\includegraphics[height=1.25in,width=1in]{pic_blank.png}}
  & \large George V. Neville-Neil \\\\
  & \large gnn@neville-neil.com \\
  & \large www.neville-neil.com \\
\end{tabular}
\vspace{5mm}
\begin{center} Syllabus, subject to modification. \\
\end{center}

% Course details
\textbf {\large \\ Course Description:} Using the DTrace system
students explore the various sub-systems of a modern operating system,
including the scheduler, memory management, networking and storage.
Each class introduces a new topic from a basic theoretical standpoint,
and then works through several practical examples in which
applications interact with the operating system.  Students will write
scripts during their lab and homework sessions to investigate and
answer questions about how each subsystem is reacting when presented
with common applications, such as a web server or database.  The use
of tracing allows us to cover a broad range of topics in a single
week.
\\
\textbf {Prerequisite(s):} None.

\textbf {Note(s):} A minimum grade of C is required in this course to progress to COURSE. 

\textbf {Credit Hours:} 3 \\

\textbf {\large Text(s):}

\begin{itemize}
\item Marshall Kirk McKusick, George V. Neville-Neil, and Robert
  N. M. Watson\emph{The Design and Implementation of the FreeBSD
    Operating System}, 2\textsuperscript{nd} Edition. Prentice Hall
  Press, Upper Saddle River, NJ, USA, April 2011.
\item Brendan Gregg and Jim Mauro. \emph{DTrace: Dynamic Tracing in
    Oracle Solaris, Mac OS X and FreeBSD}, Prentice Hall Press, Upper
  Saddle River, NJ, USA, April 2011.
\item Brendan Gregg. \emph{Systems Performance: Enterprise and the
    Cloud}, 1st Edition, Prentice Hall, October 2013
\end{itemize}

\textbf {\large Course Objectives:} \\
At the completion of this course, students will be able to:
\begin{enumerate} \itemsep-0.4em
  \item Understand basic operating systems concepts
  \item Explore large and complex systems with tracing
  \item Explain complex behaviors found in modern software systems
\end{enumerate}

% I recommend using \newpage here if necessary
% \textbf {\large Grade Distribution:} \\
% \hspace*{40mm}
% \begin{tabular}{ l l }
% Labs & 20\% \\
% Assignments & 20\% \\
% Project & 10\% \\
% Quizzes  & 10\% \\
% Midterm Exam  & 20\% \\
% Final Exam  & 20\%
% \end{tabular} \\\\

% \textbf {\large Letter Grade Distribution:} \\\\
% \hspace*{40mm}
% \begin{tabular}{ l l | l l }
% \textgreater= 93.00 & A & 73.00 - 76.99 & C \\
% 90.00 - 92.99 & A-  & 70.00 - 72.99 & C- \\
% 87.00 - 89.99 & B+  & 67.00 - 69.99 & D+ \\
% 83.00 - 86.99 & B  & 63.00 - 66.99 & D \\
% 80.00 - 82.99 & B-  & 60.00 - 62.99 & D- \\
% 77.00 - 79.99 & C+  & \textless= 59.99 & F \\
% \end{tabular} \\

% Course Policies. These are just examples, modify to your liking.
\textbf {\large Course Policies:}
\begin{itemize}
% \item \textbf {General}
%   \begin{itemize}
%   \item Computers are not to be used unless instructed to do so.
%   \item Quizzes and exams are closed book, closed notes.
%   \item \textbf {No makeup quizzes or exams will be given.}
%   \end{itemize}
% \item \textbf {Grades}
%   \begin{itemize}
%   \item Grades in the \textbf{C} range represent performance that \textbf{meets expectations}; Grades in the \textbf{B} range represent performance that is \textbf{substantially better} than the expectations; Grades in the \textbf{A} range represent work that is \textbf{excellent}.
%   \item Grades will be maintained in the LMS course shell. Students are responsible for tracking their progress by referring to the online gradebook.
%   \end{itemize}
\item \textbf {Labs and Assignments}
  \begin{itemize}
  \item Students are expected to work independently. \textbf{Offering}
    and \textbf{accepting} solutions from others is an act of
    \textbf{plagiarism}, which is a serious offense and \textbf{all
      involved parties will be penalized according to the Academic
      Honesty Policy}. Discussion amongst students is encouraged, but
    when in doubt, direct your questions to the professor, tutor, or
    lab assistant.
  \item \textbf{No late assignments will be accepted under any circumstances}.
  \end{itemize}
% \item \textbf{Attendance and Absences}
%   \begin{itemize}
%   \item Attendance is expected and will be taken each class. You are
%     allowed to miss \textbf{1} class during the semester without
%     penalty. Any further absences will result in point and/or grade
%     deductions.
%   \item Students are responsible for all missed work, regardless of
%     the reason for absence. It is also the absentee's responsibility
%     to get all missing notes or materials.
%  \end{itemize}
\end{itemize}

% College Policies
%\textbf {\large Academic Honesty Policy Summary:} 
% This should be specific to your instituition, an example is provided.

\newpage

% Course Outline
\textbf {\large Course Outline}:

The coverage might change as it depends on the progress of the class.
However, you must keep up with the reading assignments.

\begin{table}[h!]
\normalsize % The size of the table text can be changed depending on content. Remove if desired.
\begin{tabular}{ | c | c | }
\hline
\textbf{Class} & \textbf{Content} \\
\hline
  Class 1 & \begin{minipage}{.85\textwidth}
    \begin{itemize} \itemsep-0.4em
      \vspace{1mm}
    \item Course Overview
    \item Introduction to Tracing
    \item Lab 0: System Setup and Testing
      \vspace{1mm}
    \end{itemize}
  \end{minipage} \\
  \hline
  Class 2 & \begin{minipage}{.85\textwidth}
    \begin{itemize} \itemsep-0.4em
      \vspace{1mm}
    \item Tracing Tools
    \item Lab 1: Getting started with Tracing
      \vspace{1mm}
    \end{itemize}
  \end{minipage} \\
  \hline
  Class 3 & \begin{minipage}{.85\textwidth}
    \begin{itemize} \itemsep-0.4em
      \vspace{1mm}
    \item The Process Model
    \item Structure of a Process
    \item Threads vs. Processes
    \item Lab 2: Tracing Processes
      \vspace{1mm}
    \end{itemize}
  \end{minipage} \\
  \hline
  Class 4 & \begin{minipage}{.85\textwidth}
    \begin{itemize} \itemsep-0.4em
      \vspace{1mm}
    \item The Process Model 
    \item Controlling Processes
    \item Signals
    \item Lab 3: Tracking Signals
      \vspace{1mm}
    \end{itemize}
  \end{minipage} \\
  \hline
  Class 5 & \begin{minipage}{.85\textwidth}
    \begin{itemize} \itemsep-0.4em
      \vspace{1mm}
    \item Memory for Programs
    \item A Process's view of Memory
    \item Virtual Memory and the Illusion of Infinite Space
    \item Lab 4: Memory Allocations
      \vspace{1mm}
    \end{itemize}
  \end{minipage} \\
  \hline
  Class 6 & \begin{minipage}{.85\textwidth}
    \begin{itemize} \itemsep-0.4em
      \vspace{1mm}
    \item Memory for the OS
      \vspace{1mm}
    \end{itemize}
  \end{minipage} \\
  \hline
  Class 7 & \begin{minipage}{.85\textwidth}
    \begin{itemize} \itemsep-0.4em
      \vspace{1mm}
    \item Inter-Process Communication
    \item Sockets API and Local IPC
    \item Lab 5: Sockets
      \vspace{1mm}
    \end{itemize}
  \end{minipage} \\
  \hline
  Class 8 & \begin{minipage}{.85\textwidth}
    \begin{itemize} \itemsep-0.4em
      \vspace{1mm}
    \item Network Communication: Layering Concepts
    \item Unreliable Communication
      \vspace{1mm}
    \end{itemize}
  \end{minipage} \\
  \hline
  Class 9 & \begin{minipage}{.85\textwidth}
    \begin{itemize} \itemsep-0.4em
      \vspace{1mm}
    \item Networking: Reliable Communication
    \item Lab 6: Tracing TCP
      \vspace{1mm}
    \end{itemize}
  \end{minipage} \\
  \hline
  Class 10 & \begin{minipage}{.85\textwidth}
    \begin{itemize} \itemsep-0.4em
      \vspace{1mm}
    \item Data Storage: The Illusion of Names
    \item Lab 7: Naming Names
      \vspace{1mm}
    \end{itemize}
  \end{minipage} \\
  \hline
  Class 11 & \begin{minipage}{.85\textwidth}
    \begin{itemize} \itemsep-0.4em
      \vspace{1mm}
    \item Data Storage: Filesystem Organization
    \item Lab 8: VNODE Operations
      \vspace{1mm}
    \end{itemize}
  \end{minipage} \\
  \hline
  Class 12 & \begin{minipage}{.85\textwidth}
    \begin{itemize} \itemsep-0.4em
      \vspace{1mm}
    \item Data Storage: Blocks of Bits
    \item Lab 9: Where is that data?
      \vspace{1mm}
    \end{itemize}
  \end{minipage} \\
  \hline
  Class 13 & \begin{minipage}{.85\textwidth}
    \begin{itemize} \itemsep-0.4em
      \vspace{1mm}
    \item Whole System Interactions
      \vspace{1mm}
    \end{itemize}
  \end{minipage} \\
  \hline
  Class 14 & \begin{minipage}{.85\textwidth}
    \begin{itemize} \itemsep-0.4em
      \vspace{1mm}
    \item Final Exam
      \vspace{1mm}
    \end{itemize}
  \end{minipage} \\
  \hline
\end{tabular} 
\end{table}

\end{document}

%%% Local Variables:
%%% mode: latex
%%% TeX-master: t
%%% End:
