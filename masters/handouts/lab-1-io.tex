\documentclass[a4paper,10pt]{article}
\usepackage{fullpage}
\usepackage{times}
\usepackage{url}

\begin{document}

\title{L41: Lab 1 - Getting Started with Kernel Tracing - I/O}
\author{Dr Robert N.M. Watson}
\date{Michaelmas Term 2015}
\maketitle

%
% Key methodology: how to trace stuff with DTrace; probe effect; benchmark
% reproducibility; statistical significance.
%
% Key kernel concepts: system calls, other traps, kernel memory allocation,
% (perhaps) kernel threads, I/O behaviour, different types of linking, the
% overhead imposed by filesystems.
%

\noindent
The goals of this lab are to:

\begin{itemize}
\item Introduce you to our experimental environment and DTrace
\item Have you explore user-kernel interactions via system calls and traps
\item Gain experience tracing I/O behaviour in UNIX
\item Build intuitions about the probe effect
\end{itemize}

\noindent
You will do this by using DTrace to analyse the behaviour of a potted,
kernel-intensive block-I/O benchmark.

%\textit{NOTE: We require a small disk-centered I/O benchmark that allows the
%  I/O block size to be configured, in both static vs dynamic forms, that can
%  work on both a file and a raw partition.  DTrace must be able to track both
%  static and FBT probes.  We should come up with a list of the probe types of
%  interest and make sure we teach a bit about them in lecture.  We should
%  also ensure we teach about the relevant DTrace features -- e.g., execname,
%  various stats aggregation features, etc.}

\section*{Background: POSIX I/O system calls}

POSIX defines a number of synchronous I/O APIs, including the
\texttt{read()} and \texttt{write()} system calls, which accept a file
descriptor to a file or other storage object, a pointer to a buffer to read to
or write from, and a buffer length as arguments.
Although they are synchronous calls, some underlying aspects of I/O are
frequently deferred; for example, \texttt{write()} will store changes in the
buffer cache, but they may not make it to disk for some time unless a call is
made to \texttt{fsync()} to force a writeback.

You may wish to read the FreeBSD \texttt{read(2)}, \texttt{write(2)}, and
\texttt{fsync(2)} system-call manual pages to learn more about the calls
before proceeding with this lab.
You can do this using the \texttt{man} command; you may need to specify the
\textit{manual section} to ensure that you get API documentation rather than
program documentation; for example, \texttt{man 2 write} for the system call
to ensure that you don't get the \texttt{write(1)} program man page.

\section*{The benchmark}

Our I/O benchmark is straightforward: it performs a series of \texttt{read()}
or \texttt{write()} I/O system calls in a loop using configurable buffer and
total I/O sizes.
Optionally, an \texttt{fsync()} system call be issued at the end of the I/O
loop to ensure that buffered data is written to disk.
The benchmark samples the time before and after the I/O loop, optionally
displaying an average bandwidth.
The lab bundle will build two versions of the benchmark: \texttt{io-static}
and \texttt{io-dynamic}.
The former is statically linked (i.e., without use of the run-time linker or
dynamic libraries), whereas the latter is dynamically linked.

\subsection*{Compiling the benchmark}

You will need to copy the lab bundle to your BeagleBone Black, untar it, and
build it before you can begin work.
Once you have configured the BBB so that you can log in (see \textit{L41: Lab
Setup}), you can use SSH to copy the bundle to the BBB:

\begin{verbatim}
scp /anfs/www/html/teaching/1516/L41/labs/io.tgz guest@192.168.141.100:/data
\end{verbatim}

\noindent
Logged in as \texttt{guest}, untar and build the bundle:

\begin{verbatim}
cd /data
tar -xzf io.tgz
cd io
make
\end{verbatim}

\subsection*{Running the benchmark}

Once built, you can run the benchmark binaries as follows, with command-line
arguments specifying various benchmark parameters:

\begin{verbatim}
./io-static
\end{verbatim}

\noindent
or:

\begin{verbatim}
./io-dynamic
\end{verbatim}

The benchmark must be run in one of three operational modes: \textit{create},
\textit{read}, or \textit{write}, specified by flags.
In addition, the target file must be specified.
If you run the \texttt{io-static} or \texttt{io-dynamic} benchmark without
arguments, a small usage statement will be printed, which will also identify
the default buffer and total I/O sizes configured for the benchmark.

In your experiments, you will need to be careful to hold most variables
constant in order to isolate the effects of specific variables; for example,
you may wish to hold the total I/O size constant as you vary the buffer size.
You may wish to experiment initially using \url{/dev/zero} -- the kernel's
special device node providing an unlimited source of zeros, but will also want
to run the benchmark against a file in the filesystem.

\subsection*{Required operation flags}

Specify the mode in which the benchmark should operate:

\begin{description}
\item[-c] Create the specified file using the default (or requested) total
  I/O size.

\item[-r] Benchmark \texttt{read()} of the target file, which must already
  have been created.

\item[-w] Benchmark \texttt{write()} of the target file, which must already
  have been created.
\end{description}

\subsection*{Optional I/O flags}

\begin{description}
\item[-b \textit{buffersize}] Specify an alternative buffer size in bytes.
  The total I/O size must be a multiple of buffer size.

\item[-t \textit{totalsize}] Specify an alternative total I/O size in bytes.
  The total I/O size must be a multiple of buffer size.

\item[-B] \textit{Bare mode} disables the normal preparatory work done by the
  benchmark, such as flushing outstanding disk writes and sleeping for one
  second.
  This mode is preferred when performing whole-program benchmarking.

\item[-d] \textit{Direct I/O mode} disables use of the buffer cache by the
  benchmark by specifying the \texttt{O\_DIRECT} flag to the \texttt{open()}
  system call on the target file.
  When switching from buffered mode, the first measurement using \texttt{-d}
  should be discarded, as some cached data may still be used.

\item[-s] \textit{Synchronous mode} causes the benchmark to call
  \texttt{fsync()} after writing the file to cause all buffered writes to
  complete before the benchmark terminates -- and in particular before the
  final timestamp is taken.
\end{description}

\subsection*{Terminal output flags}

\noindent
The following arguments control terminal output from the benchmark; remember
that output can substantially change the performance of the system under test,
and so you should ensure that output is either entirely suppressed during
tracing and benchmarking, or that tracing and benchmarking only occurs during
a period of program execution unaffected by terminal I/O:

\begin{description}
\item[-q] \textit{Quiet mode} suppress all terminal output from the benchmark,
  which is preferred when performing whole-program benchmarking.

\item[-v] \textit{Verbose mode} causes the benchmark to print additional
  information, such as the time measurement, buffer size, and total I/O size.
\end{description}

\subsection*{Example benchmark commands}

This command creates a default-sized data file in the \url{/data} filesystem:

\begin{verbatim}
./io-static -c /data/iofile
\end{verbatim}

\noindent
This command runs a simple \texttt{read()} benchmark on the data file,
printing additional information about the benchmark run:

\begin{verbatim}
./io-static -v -r /data/iofile
\end{verbatim}

\noindent
This command runs a simple \textit{write()} benchmark on the data file,
printing additional information about the benchmark run:

\begin{verbatim}
./io-static -v -w /data/iofile
\end{verbatim}

\noindent
If performing whole-program analysis using DTrace, be sure to suppress output
and run the benchmark in bare mode:

\begin{verbatim}
./io-static -B -q -r /data/iofile
\end{verbatim}

\noindent
The following command disables use of the buffer cache when running a read
benchmark; be sure to discard the output of the first run of this command:

\begin{verbatim}
./io-static -d -r /data/iofile
\end{verbatim}

\noindent
To better understand kernel behaviour, you may also wish to run the benchmark
against \url{/dev/zero}, a pseudodevice that returns all zeroes, and discards
all writes:

\begin{verbatim}
./io-static -r /dev/zero
\end{verbatim}

\noindent
To get a high-level summary execution time, including a breakdown of total
wall-clock time, time in userspace, and `system time', use the UNIX
\texttt{time} command:

\begin{verbatim}
/usr/bin/time -p ./io-static -r -B -d -q /data/iofile
real 1.31
user 0.00
sys 0.31
\end{verbatim}

\noindent
However, it may be desirable to use DTrace to collect more granular timestamps
by instrumenting return from \texttt{execve()} and entry of \texttt{exit()}
for the program under test.

This run of \texttt{io-static}, reading the data file \url{/data/iofile},
bypassing the buffer cache, and running in bare mode (i.e., without quiescing
prior to the benchmark) took 1.31 seconds of wall-clock time to complete.
Of this, (roughly) 0.00 seconds were spent in userspace, and (roughly) 0.31
seconds were spent in the kernel on behalf of the process.
From this output, it is unclear where the remaining 1.00 seconds were spent,
but presumably a substantial fraction was spent blocked on (slow) SD Card I/O.
Time may also have been spent running other processes -- and lower-precision
time measurement, such as provided by \texttt{time}, may suffer from
non-trivial rounding error.

\section*{Objects on which to run the benchmark}

You may wish to point the benchmark at one of two objects:

\begin{description}
\item[\url{/dev/zero}] The zero device: an infinite source of zeroes, and also
  an infinite sink for any data you write to it.

\item[\url{/data/iofile}] A writable file in a journalled UFS filesystem on
  the SD card.
  You will need to create the file using \texttt{-c} before performing
  \texttt{read()} or \texttt{write()} benchmarks.
\end{description}

\section*{Notes on using DTrace}

You will want to use DTrace in two ways during this lab:

\begin{itemize}
\item To analyse just the I/O loop itself; or
\item To analyse whole-program execution including setup, run-time linking,
  etc.
\end{itemize}

In the former case, it is useful to know that the system call
\texttt{clock\_gettime} is both run immediately before, and immediately after,
the I/O loop.
You may wish to bracket tracing between a return probe for the former, and an
entry probe for the latter.
For example, you might wish to include the following in your DTrace scripts:

\begin{verbatim}
syscall::clock_gettime:return
/execname == "io-static" && !self->in_benchmark/
{
    self->in_benchmark = 1;
}

syscall::clock_gettime:entry
/execname == "io-static" && self->in_benchmark/
{
    self->in_benchmark = 0;
}

END
{
    /* You might print summary statistics here. */
    exit(0);
}
\end{verbatim}

\noindent
Other DTrace predicates can then refer to \texttt{self->in\_benchmark} to
determine whether the probe is occuring during the I/O loop.
The benchmark is careful to set up the runtime environment suitably before
performing its first clock read, and to perform terminal output only after the
second clock read, so it is fine to leave benchmark terminal output enabled.

In the latter case, you will wish to configure the benchmark for whole-program
analysis by disabling all output (\texttt{-q}) and enabling bare mode
(\texttt{-B}).
In this case, you will simply want to match the executable name for system
calls or traps using \texttt{/execname == "io-static"/} (or
\texttt{io-dynamic} as required).

\section*{Notes on using \texttt{ministat}}

While more sophisticated statistics and graphing packages such as \textbf{R}
may be appropriate for preparing your lab report, they can be rather
heavyweight during casual performance investigation.
You may wish to use the command-line \texttt{ministat} tool to analyse simple
distributions and perform statistical tests.
\texttt{ministat} accepts file arguments consisting of a list of numeric
values, one per line, and generates text output to the terminal.
Different data sets passed via multiple file arguments will be compared using
Student's \textit{t}-test.

\section*{Notes on benchmark}

This benchmark calculates average I/O rate for a particular run.
Be sure to run the benchmark more than once (ideally, perhaps a dozen times),
and discard the first output which may otherwise be affected by prior
benchmark runs (e.g., if data is left in the buffer cache and the benchmark is
not yet in the steady state).
Do be sure that terminal I/O from the benchmark is not included in tracing or
time measurements.

\section*{Exploratory questions}

These questions are intended to help you understand the behaviour of the I/O
benchmark, and may provide supporting evidence for your experimental
questions.
However, they are just suggestions -- feel free to approach the problem
differently!

\begin{enumerate}
  \item Baseline benchmark performance analysis:
  \begin{itemize}
    \item How do \texttt{read()} and \texttt{write()} performance compare?
    \item Describe the performance effect of using the buffer cache?
      Be sure to consider the effects of both \texttt{-d} and \texttt{-s}.
    \item What proportion of benchmark time is spent in userspace vs. kernel?
    \item How many times are different system calls called during the I/O
      loop?
    \item How much time is spent, cumulatively, in each system call by type
      during the I/O loop?
    \item What is the role of traps in execution of the I/O loop?
    \item How does work performed in just the I/O loop compare with
      whole-program behaviour?
  \end{itemize}
  \item Probe effect and measurement decisions
  \begin{itemize}
    \item How does performance change if you insert system-call or trap probes
      in the I/O loop?
    \item What sources of variance may be affecting benchmark performance,
      and how can we measure them?
  \end{itemize}
\end{enumerate}

\section*{Experimental questions}

Your lab report will compare several configurations of the benchmark,
exploring (and explaining) performance differences between them.
Do ensure that your experimental setup suitably quiesces other activity on the
system, and also use a suitable number of benchmark runs; you may wish to
consult the \textit{FreeBSD Benchmarking Advice} wiki page linked to from the
module's reading list for other thoughts on configuring the benchmark setup.
% NB: No longer assigned for lab 1.
%The research reading by Ellard and Seltzer (2003) may also offer insight into
%benchmark considerations.
The following questions are with respect to the benchmark reading a file
through the buffer cache:

\begin{itemize}
  \item Holding the total I/O size constant (16MB), how does varying I/O
    buffer size affect IO-loop performance?
  \item Holding the buffer size constant (16K) and varying total I/O size, how
     does static vs. dynamic linking affect whole-program performance?
%    Measure whole-program execution time using a DTrace script that captures
%    timestamps on return from \texttt{execve()} and entry to \texttt{exit()}
%    for \texttt{io-static} and \texttt{io-dynamic}.
  \item At what file-size threshold does the performance difference between
    static and dynamic linking fall below 5\%?  At what file-size threshold
    does the performance difference fall below 1\%?
\end{itemize}

\noindent
For the purposes of performance graphs, measured performance (the dependent
variable, or Y axis) is with respect to I/O bandwidth rather than literal
execution time.
This will allow us to analyse relative efficiency as file and buffer sizes
vary.

\end{document}
