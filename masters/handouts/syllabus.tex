\documentclass[a4paper,10pt]{article}
\usepackage{fullpage}
\usepackage{times}

\begin{document}
\title{L41: Advanced Operating Systems - Syllabus}
\author{Dr Robert N.M. Watson}
\date{Michaelmas Term 2015}
\maketitle

\noindent
\begin{description}
\item[Code:] L41
\item[Instructor:] Dr Robert N. M. Watson
\item[Prerequisites:] Undergraduate operating-systems course; please see below
  for further details
\item[Structure:] Six 1-hour lectures; five 2-hour labs
\end{description}

\section{Aims}

\textit{Systems research} refers to the study of a broad range of behaviours
arising from complex system design, including low-level operating systems,
resource sharing and scheduling, interactions between hardware and software,
network-protocol design and implementation, separation of mutually distrusting
parties on a common platform, and control of distributed-system behaviours
such as concurrency and data replication.
This module will:

\begin{enumerate}
\item Teach systems-analysis methodology and practice through tracing and
  performance profiling experiments;
\item Expose students to real-world systems artefacts such as the
  OS scheduler, Inter-Process Communication (IPC), and network stack, and
  consider their hardware-software interactions with storage devices and CPUs;
\item Develop scientific writing skills through a series of laboratory
  reports; and
\item Assign a selection of original research papers to give insight into
  potential research topics and approaches.
\end{enumerate}

The teaching style will blend lectures and hands-on labs that teach
methodology, design principles, and practical skills.
Students will be taught about (and assessed via) a series of lab-report-style
assignments based on in- and out-of-classroom practical work.
The systems studied are real, and all wires will be live.

\section{Prerequisites}

It is strongly recommended that students:

\begin{enumerate}
\item Have previously (and successfully) completed an undergraduate
  operating-system course with practical content -- or have equivalent
  experience through project or open-source work.
\item Have reasonable comfort with the C and Python programming languages.
  C is the primary implementation language for systems that we will analyse,
  requiring reading fluency; userspace C programs will also be written and may
  be extended as part of lab exercises.
  Python may prove useful as a data-processing language, and provides useful
  tools for data analysis and presentation.
%  Students without a Python background will wish to complete an online Python
%  tutorial prior to term, as the language will be used in data collection,
%  analysis, and presentation from the first lab.
\item Review an undergraduate OS textbook (such as Silberschatz, et al.) to
  ensure that basic OS concepts such as the \textit{process model},
  \textit{Inter-Process Communication}, \textit{filesystems}, and
  \textit{virtual memory} are familiar.
\item Be comfortable with the UNIX command-line environment including
  compiler/debugging tools.
  Students without this background may wish to sit in on the undergraduate
  UNIX Tools lecture series in Michaelmas.
\end{enumerate}

\section{Lectures, Labs, and Lab Reports}

\begin{description}
\item[Submodule 1: Introduction to kernels and tracing/analysis]
  The purpose of this submodule is to introduce students to the structure of
  a contemporary operating-system kernel through tracing and profiling.
  The first lab report is written, which will be receive feedback from the
  instructor, but not contribute to the final mark.

  \noindent
  \textbf{Lecture 1: Introduction: OSes, Systems Research, and L41 (1h)}

  The first lecture reintroduces the idea of an operating system, its
  role, contemporary operating-system structure, and current operating-system
  research areas and venues.
  We will also discuss how (and why) operating systems are taught, and the
  approach taken in this module.

  \noindent
  \textbf{Lecture 2: Kernels and Tracing (1h)}

  The second lecture continues our exploration of OS structure.
  We look at the goals, implementation, potential uses of DTrace as a means of
  kernel instrumentation and tracing, and the probe effect.
  We also consider the high-level structure of a kernel (is it just a complex
  C program?) and its execution model.

%  Students will gain familiarity with practical systems tools such as DTrace
%  and hardware performance counters, as well as with data interpretation,
%  analysis, and presentation using Python and GraphViz.

  \noindent
  \textbf{Lab 1: Getting Started with Kernel Tracing - I/O (2h)}

  The first lab uses an exploration of POSIX file I/O to motivate learning
  about DTrace, user-kernel interactions, and performance analysis.
  The first lab report will describe this lab.

  \noindent
  \textbf{Deliverable: Lab Report 1 - POSIX I/O Performance Analysis}

\item[Submodule 2: The Process Model]
  This submodule introduces students to concrete implications of the UNIX
  process model: processes and threads in both userspace and kernelspace, the
  hardware foundations for kernel and process isolation, system calls, and
  traps.
  The second lab report will be written.

  \noindent
  \textbf{Lecture 3: The Process Model - 1 (1h)}
  The third lecture looks at the evolution of the process model, from its
  1960s origins to the 1990s deployment of ELF, dynamic linking, and
  multithreading in UNIX.
  We take an initial dive into virtual memory, as well as the hardware
  foundations for system calls and traps.

  \noindent
  \textbf{Lecture 4: The Process Model - 2 (1h)}
  The fourth lecture continues our discussion of system calls and traps.
  We consider their semantics, the system-call table and surrounding software
  stack, and their (desirable) security properties.
  We also begin to explore the implied (and very real) cost of the process
  model itself, revisiting virtual memory through insights from the Mach
  project.

  \noindent
  \textbf{Lab 2: Kernel Implications of IPC (2h)}
  The second lab uses DTrace to understand the dynamics of local
  Inter-Process Communication: kernel memory allocation, copying, locking,
  scheduling, and message-based IPC.
  Of particular concern will be building an understanding of basic IPC
  functionality, but also of how it interacts with buffering and the scheduler
  to affect IPC latency and throughput.

  \noindent
  \textbf{Lab 3: Micro-Architectural Implications of IPC (2h)}
  The third lab introduces a new performance analysis mechanism,
  \textit{hardware performance counters} that allow direct monitoring of
  low-level architectural and micro-architectural details of performance.
  Using this tool, we will revisit existing benchmarks to explain the use of
  CPU time by the application and kernel.

  \noindent
  \textbf{Deliverable: Lab Report 2 - Inter-Process Communication Performance}

\item[Submodule 3: TCP/IP]
  This submodule introduces a contemporary network stack, with a particular
  interest in the TCP protocol.
  Labs will consider both the behaviour of TCP connections, exploring the TCP
  state machine, socket-buffer interactions with flow control, and congestion
  control.
  Students will use Dummynet to simulate network latency and explore how TCP
  slow start and congestion avoidance respond to network conditions.
  The third (and final) lab report will be written.

  \noindent
  \textbf{Lecture 5: The Network Stack - 1 (1h)}
  The fifth lecture introduces the history and role of networking in OS
  design, with a focus on the Berkeley Sockets API and TCP/IP stack.
  We explore the flow of memory both from the perspective of hardware (NIC,
  DMA, memory, caches, and processor) and software (applications, buffering,
  the protocol stack, memory allocator, and device driver).
  We consider the input, output, and forwarding paths, with an interest in
  dispatch models and their interaction with multiprocessor systems.
  Finally, we look at two recent pieces of network-stack research, Netmap and
  network-stack specialisation.

  \noindent
  \textbf{Lecture 6: The Network Stack - 2 (1h)}
  The final lecture explores TCP protocol and implementation behaviour in
  greater detail.
  We consider the objectives and evolution of TCP, especially with respect to
  congestion control, performance, and denial-of-service (DoS) resistance.
  We also explore the evolution of in-kernel data structures in network-stack
  scalability.
  Research topics include the development of congestion control and differing
  models for network-stack multiprocessing.
  Finally, we consider how changes in NIC, bus, and processor hardware have
  impacted (and continue to affect) the implementation of TCP.

  \noindent
  \textbf{Lab 4: The TCP State Machine (2h)}
  The fourth lab asks students to explore the TCP state machine in practice:
  how it is triggered by both API and network-level events.
  An early measurement of the impacts of network latency on TCP is performed.

  \noindent
  \textbf{Lab 5: TCP Latency and Bandwidth (2h)}
  The final lab continues our investigation of the effects of network latency
  on TCP performance, and especially its interactions with congestion-control
  slow start and steady state.
  We also explore how socket-buffer configuration affects flow control, and
  the combined end effects on available bandwidth.

  \noindent
  \textbf{Deliverable: Lab Report 3 - The TCP State Machine, Latency, and
    Bandwidth}
\end{description}

\section{Objectives}

On completion of this module, students should:

\begin{itemize}
\item Have an understanding of high-level OS kernel structure
\item Gained insight into hardware-software interactions for compute and I/O
\item Have practical skills in system tracing and performance analysis
\item Have been exposed to research ideas in system structure and behaviour
\item Have learned how to write systems-style performance evaluations
\end{itemize}

\section{Coursework}

Students will write and submit three lab reports to be marked by the
instructor.
The first report is a `practice run' intended to help students develop 
analysis techniques and writing styles, and will not contribute to the final
mark.
The remaining two reports are marked and assessed, each constituting 50\% of
the final mark.
Class participants are advised that, in prior teaching of this module, the
average final mark of students submitting the first lab report was
significantly greater than those who did not.
% 6% in case anyone is interested -- often the difference between being above
% or below the distinction mark.

\section{Practical work}

Five 2-hour in-classroom labs, supplemented by follow-up work outside of the
class itself, will ask students to develop and use skills in tracing and 
performance analysis as applied to real-world systems artefacts.
Results from these labs will be the primary input to lab reports.
Please see the handout, \textit{L41: Lab Setup}, for details on the lab
platform.
Typical labs will involve using tracing and profiling to characterise specific
behaviours (e.g., file I/O in terms of system calls and traps) to diagnose
application-level behaviours (e.g., with respect to effective use of TCP
sockets for bulk data transport).
Students may find it useful to work in pairs within the lab, but must prepare
lab reports independently.
The module lecturer will give a short introductory lecture at the start of
each lab, and instructors will be on-hand throughout labs to provide
assistance.
Lab participation is not directly included in the final mark, but lab work is
a key input to lab reports, which are assessed.

\section{Assessment}

Please see the handout, \textit{L41: Lab Reports}, for a description of the
lab-report format and its assessment.

\section{Recommended reading}

Please see the handout, \textit{L41: Readings}, for a list of module texts,
research readings, and supplemental texts.

\end{document}
