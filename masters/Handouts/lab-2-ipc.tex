\documentclass[a4paper,10pt]{article}
\usepackage{fullpage}
\usepackage{times}
\usepackage{url}

\begin{document}

\title{L41: Lab 2 - Kernel Implications of IPC}
\author{Dr Robert N.M. Watson}
\date{Michaelmas Term 2015}
\maketitle

%
% Key methodology: how to trace stuff with DTrace; probe effect; benchmark
% reproducibility; statistical significance.
%
% Key kernel concepts: system calls, other traps, kernel memory allocation,
% (perhaps) kernel threads, I/O behaviour, different types of linking, the
% overhead imposed by filesystems.
%

\noindent
The goals of this lab are to:

\begin{itemize}
\item Continue to gain experience tracing user-kernel interactions via system calls
\item Explore the performance of varying IPC models, buffer sizes, and process models
% \item Measure the probe effect and explore mitigation techniques such as sampling
\item Gather data to support writing your first assessed lab report
\end{itemize}

\noindent
You will do this by using DTrace to analyse the behaviour of a potted,
kernel-intensive IPC benchmark.

\section*{Background: POSIX IPC objects}

POSIX defines several types of Inter-Process Communication (IPC) objects,
including pipes (created using the \texttt{pipe()} system call) and sockets
(created using the \texttt{socket()} and \texttt{socketpair()} system calls).
Pipes are used most frequently between pairs of processes in a UNIX
\textit{process pipeline}: a chain of processes started by a single command
line, whose output and input file descriptors are linked.
Although pipes can be set up between unrelated processes, the primary means of
acquiring a pipe is through inheritence across \texttt{fork()}, meaning that
they are used between closely related processes (e.g., with a common parent
process).
Sockets are used when two processes are created in independent contexts and
must later rendezvous -- e.g., via the filesystem, but also via TCP/IP.
In typical use, each endpoint process creates a socket via the
\texttt{socket()} system call, which are then interconnected through use of
\texttt{bind()}, \texttt{listen()}, \texttt{connect()}, and \texttt{accept()}.
However, there is also a \texttt{socketpair()} system call that returns a pair
of interconnected endpoints in the same style as \texttt{pipe()} -- convenient
for us as we wish to compare the two side-by-side.

Both pipes and sockets can be used to transmit ordered byte streams: a
sequence of bytes sent via one file descriptor that will be received reliably
on the other without loss or reordering.
As file I/O, the \texttt{read()} and \texttt{write()} system calls can be used
to read and write data on file descriptors for pipes and sockets.
It is useful to know that these system calls are permitted to return
\textit{partial reads} and \textit{partial writes}: i.e., a buffer of some
size (e.g., 1k) might be passed as an argument, but only a subset of the
requested bytes may be received or sent, with the actual size returned via the
system call's return value.
This may happen if the in-kernel buffers for the IPC object are too small for
the full amount, or if \textit{non-blocking I/O} is enabled.
When analysing traces of IPC behaviour, it is important to consider both the
size of the buffer passed and the number of bytes returned in evaluating the
behaviour of the system call.

You may wish to read the FreeBSD \texttt{pipe(2)} and \texttt{socketpair(2)}
manual pages to learn more about the calls before proceeding with this lab.

\section*{The benchmark}

As with our earlier I/O benchmark, the IPC benchmark is straightforward: it
sets up a pair of IPC endpoints referencing a shared pipe or socket, and then
performs a series of \texttt{write()} and \texttt{read()} system calls on the
file descriptors to send (and then receive) a total number of bytes of data.
Data will be sent using a smaller userspace buffer size -- although as hinted
above, there is no guarantee that a full user buffer will be sent or received
in any individual call.
As with the earlier I/O benchmark, there are several modes of operation:
sending and receiving within a single thread, a pair of threads in the same
process, or between two threads in two different processes.
The benchmark will set up any necessary IPC objects, threads, and processes,
sample the start time using the \texttt{clock\_gettime()} system call, perform
the IPC loop (perhaps split over two threads), and then sample the finish time
using the \texttt{clock\_gettime()} system call.
Optionally, both the average bandwidth across the IPC object, and also more
verbose information about the benchmark configuration, may be displayed.
Both statically and dynamically linked versions of the binary are provided:
\texttt{ipc-static} and \texttt{ipc-dynamic}.


\subsection*{Compiling the benchmark}

You will need to copy the lab bundle to your BeagleBone Black, untar it, and
build it before you can begin work.
Once you have configured the BBB so that you can log in (see \textit{L41: Lab
Setup}), you can use SSH to copy the bundle to the BBB:

\begin{verbatim}
scp /anfs/www/html/teaching/1516/L41/labs/ipc.tgz guest@192.168.141.100:/data
\end{verbatim}

\noindent
Or via the web:

\noindent
\url{https://www.cl.cam.ac.uk/teaching/1516/L41/labs/ipc.tgz}

\noindent
Logged in as \texttt{guest}, untar and build the bundle:

\begin{verbatim}
cd /data
tar -xzf ipc.tgz
cd ipc
make
\end{verbatim}

\subsection*{Running the benchmark}

Once built, you can run the benchmark binaries as follows, with command-line
arguments specifying various benchmark parameters:

\begin{verbatim}
./ipc-static
./ipc-dynamic
\end{verbatim}

If you run the benchmark without arguments, a small usage statement will be
printed, which will also identify the default IPC object type, IPC buffer, and
total IPC sizes configured for the benchmark.
As in the prior lab, you will wish to be careful to hold most variables
constant in order to isolate the effects of specific variables.
For example, you might wish the vary the IPC object type while holding the
total IPC size constant.

\subsection*{Required operation argument}

Specify the mode in which the benchmark should operate:

\begin{description}
\item[1thread] Run the benchmark entirely within one thread; note that, unlike
  other benchmark configurations, this mode interleaves the IPC calls and must
  place the file descriptors into non-blocking mode or risk deadlock.
  This may have observable effects on the behaviour of the system calls with
  respect to partial reads or writes.

\item[2thread] Run the benchmark between two threads within one process: one
  as a `sender' and the other as a `receiver', with the sender capturing the
  first timestamp, and the receiver capturing the second.
  System calls are blocking, meaning that if the in-kernel buffer fills
  during a \texttt{write()}, then the sender thread will sleep; if the
  in-kernel buffer empties during a \texttt{read()}, then the receiver thread
  will sleep.


\item[2proc] As with the \texttt{2thread} configuration, run the benchmark in
  two threads -- however, those threads will be in two different processes.
  The benchmark creates a second process using \texttt{fork()} that will run
  the sender.
  System calls in this variation are likewise blocking.
\end{description}

\subsection*{Optional I/O flags}

\begin{description}
\item[-b \textit{buffersize}] Specify an alternative userspace IPC buffer size
  in bytes -- the amount of memory allocated to hold to-be-sent or received
  IPC data.
  The same buffer size will be used for both sending and receiving.
  The total IPC size must be a multiple of buffer size.

\item[-i \textit{ipctype}] Specify the IPC object to use in the benchmark:
  \texttt{pipe} or \texttt{local}.

\item[-t \textit{totalsize}] Specify an alternative total IPC size in bytes.
  The total IPC size must be a multiple of userspace IPC buffer size.

\item[-B] Run in \texttt{bare mode}: disable normal quiescing activities such
  as using \texttt{sync()} to cause the filesystem to synchronise before the
  IPC loop runs, and using \texttt{sleep()} to await terminal-I/O quietude.
  This will be the more appropriate mode in which to perform whole-program
  analysis but may lead to greater variance if simply analysing the IPC loop.

\item[-s] When operating on a socket, explicitly set the in-kernel
  socket-buffer size to match the userspace IPC buffer size rather than using
  the kernel default.
  Note that per-process resource limits will prevent use of very large buffer
  sizes.
\end{description}

\subsection*{Terminal output flags}

\noindent
The following arguments control terminal output from the benchmark; remember
that output can substantially change the performance of the system under test,
and you should ensure that output is either entirely suppressed during tracing
and benchmarking, or that tracing and benchmarking only occurs during a period
of program execution unaffected by terminal I/O:

\begin{description}
\item[-q] \textit{Quiet mode} suppress all terminal output from the benchmark,
  which is preferred when performing whole-program benchmarking.

\item[-v] \textit{Verbose mode} causes the benchmark to print additional
  information, such as the time measurement, buffer size, and total IPC size.
\end{description}

\subsection*{Example benchmark commands}

This command performs a simple IPC benchmark using a pipe and default
userspace IPC buffer and total IPC sizes within a single thread of a single
process:

\begin{verbatim}
./ipc-static -i pipe 1thread
\end{verbatim}

\noindent
This command performs the same pipe benchmark, but between two threads of the
same process:

\begin{verbatim}
./ipc-static -i pipe 2thread
\end{verbatim}

\noindent
And this command does so between two processes:

\begin{verbatim}
./ipc-static -i pipe 2proc
\end{verbatim}

\noindent
This command performs a socket-pair benchmark, and requests non-default
socket-buffer sizes synchronised to a userspace IPC buffer size of 1k:

\begin{verbatim}
./ipc-static -i local -s -b 1024 2thread
\end{verbatim}

\noindent
As with the I/O benchmark, additional information can be requested using
\textit{verbose mode}:

\begin{verbatim}
./ipc-static -v -i pipe 1thread
\end{verbatim}

\noindent
And, likewise, all output can be suppressed, and \textit{bare mode} can be
used, for whole-program analysis:

\begin{verbatim}
./ipc-static -q -B -i pipe 1thread
\end{verbatim}

\section*{Note on kernel configuration}

By default, the kernel limits the maximum per-socket socket-buffer size that
can be configured, in order to avoid resource starvation.
You will need to tune the kernel's default limits using the following command,
run as root, prior to running benchmarks.
Note that this should be set before any benchmarks are run, whether or not
they are explicitly configuring the socket-buffer size, as the limit will also
affect socket-buffer auto-sizing.

\begin{verbatim}
sysctl kern.ipc.maxsockbuf=33554432
\end{verbatim}

\section*{Notes on using DTrace}

On the whole, this lab will be concerned with just measuring the IPC loop,
rather than whole-program behaviour.
As in the last lab, it is useful to know that the system call
\texttt{clock\_gettime} is both run immediately before, and immediately after,
the IPC loop.
In this benchmark, these events may occur in different threads or processes,
as the sender performs the initial timestamp before transmitting the first
byte over IPC, and the receiver performs the final timestamp after reciving
the last byte over IPC.
You may wish to bracket tracing between a return probe for the former, and an
entry probe for the latter; see the notes from the last lab for an example.

As with the last lab, you will want to trace the key system calls of the
benchmark: \texttt{read()} and \texttt{write()}.
For example, it may be sensible to inspect \texttt{quantize()} results for
both the execution time distributions of the system calls, and the amount of
data returned by each (via \texttt{arg0} in the system-call return probe).
You will also want to investigate scheduling events using the \texttt{sched}
provider.
This provider instruments a variety of scheduling-related behaviours, but it
may be of particular use to instrument its \texttt{on-cpu} and
\texttt{off-cpu} events, which reflect threads starting and stopping
execution on a CPU.
You can also instrument \texttt{sleep} and \texttt{wakeup} probes to trace
where threads go to sleep waiting for new data in an empty kernel buffer (or
for space to place new data in a full buffer).
When tracing scheduling, it is useful to inspect both the process ID
(\texttt{pid}) and thread ID (\texttt{tid}) to understand where events are
taking place.

In this lab, key experimental questions relate to the probe effect.
By its very nature, the probe effect is hard to investigate, as the probe
effect does, of course, affect investigation of the effect itself!
However, one simple way to approach the problem is to analyse the results of
performance benchmarking with different types of DTrace scripts running.
For example, to benchmark with more or less simple scripts for \texttt{read()}
or \texttt{write()}: simply counting calls, tracing values, or capturing
complete stack traces.
Scripts can, themselves, have stateful conditional behaviour that could limit
various types of tracing activity to, for example, $1/n$ system calls,
offering a way to mitigate certain types of overhead.
When exploring the probe effect, it is important to consider not just the
impact on bandwidth average/variance, but also on systemic behaviour: for
example, when performing more detailed tracing, causing the runtime of the
benchmark to increase, does the number of context switches increase, or the
distribution of \texttt{read()} return values?
In general, our interest will be in the overhead of probes rather than the
overhead of terminal I/O from the DTrace process -- you may wish to suppress
that output during the benchmark run so that you can focus on probe overhead.

\section*{Notes on benchmark}

As with the prior lab, it is important to run benchmarks more than once to
collect a distribution of values, allowing variance to be analysed.
You may wish to discard the first result in a set of benchmark runs as the
system will not yet have entered its steady state.
Do be sure that terminal I/O from the benchmark is not included in tracing or
time measurements (unless that is the intent).

\section*{Exploratory questions}

These questions are intended to help you understand the behaviour of the IPC
benchmark, and may provide supporting evidence for your experimental
questions.
However, they are just suggestions -- feel free to approach the problem
differently!
These questions do not need to be addressed in your lab report.

\begin{enumerate}
  \item How do the six basic modes of the benchmark compare in terms of
    IPC throughput: \{1thread, 2thread, 2proc\} $\times$ \{pipe, socket\}?
  \item How do distributions of \texttt{read()} and \texttt{write()}
    system-call return values vary from one another within one benchmark?
    Between different benchmark configurations?
  \item How does setting socket-buffer size impact performance for the
    socket configuration?
  \item How much time is spent in system calls for the pipe vs. socket
    benchmarks for the default buffer size and total IPC size?
  \item How does the number of context switches vary between the pipe and
    socket benchmarks, both with and without socket-buffer size tuning?
\end{enumerate}

\section*{Experimental questions (part 1)}

You will receive a separate handout during the next lab describing \textit{Lab
Report 2}; however, this description will allow you to begin to prepare for
the assignment, which will also depend on the outcome of the next lab.
Your lab report will compare several configurations of the IPC benchmark,
exploring (and explaining) performance differences between them.
Do ensure that your experimental setup suitably quiesces other activity on the
system, and also use a suitable number of benchmark runs; you may wish to
consult the \textit{FreeBSD Benchmarking Advice} wiki page linked to from the
module's reading list for other thoughts on configuring the benchmark setup.
The following questions are with respect to a fixed total IPC size with a
statically linked version of the benchmark, and refer only to IPC-loop, not
whole-program, analysis.
Using \texttt{2thread} and \texttt{2proc} modes, explore how varying IPC model
(pipes, sockets, and sockets with \texttt{-s}) and IPC buffer size affect
performance:

\begin{itemize}
  \item Does increasing IPC buffer size uniformly improve performance across
    IPC models?
  \item Is using multiple threads faster or slower than using multiple
    processes?
\end{itemize}

\noindent
Graphs and tables should be used to illustrate your measurement results.
Ensure that, for each question, you present not only results, but also a
causal explanation of those results -- i.e., why the behaviour in question
occurs, not just that it does.
For the purposes of graphs in this assignment, use achieved bandwidth, rather
than total execution time, for the Y axis, in order to allow you to more
directly visualise the effects of configuration changes on efficiency.

\end{document}
